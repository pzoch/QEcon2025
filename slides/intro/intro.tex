\documentclass[11pt,xcolor={dvipsnames},aspectratio=159,hyperref={pdftex,pdfpagemode=UseNone,hidelinks,pdfdisplaydoctitle=true},usepdftitle=false]{beamer}
\usepackage{presentation}[aspectratio=169]
\usepackage{math}
\usepackage{mathtools}
\usepackage{mleftright}
% Enter title of presentation PDF:
\hypersetup{pdftitle={Intro}}
% Enter link to PDF file with figures:

\begin{document}
% Enter presentation title:
\title{Introduction}
\subtitle{Quantitative Economics 2025}
% Enter presentation information:

% Enter presentation authors:
\author{Piotr Żoch}%
% Enter presentation location and date (optional; comment line if not needed):
\frame{\titlepage}



\begin{frame}{About this course}
    \begin{itemize}
    \item Goal of the course:
    \begin{itemize}
        \item Teach you tools and techniques useful in modern economics.
        \item Give you understanding of scientific computing.
        \item Prepare you for work on quantitative projects.
    \end{itemize}

    \item We will:
    \begin{itemize}
        \item Learn how to write code in Julia.
        \item Study elementary numerical methods.
        \item Apply recursive methods to economic problems.
        \item Solve and simulate economic models.
    \end{itemize}
    
    \item \alb{This course}: an \al{introduction} to the above.
    
\end{itemize}
\end{frame}



% Fill out content of presentation:
\begin{frame}{Computation in economics}
\begin{itemize} \item Computational methods are used in many fields of economics:
\begin{itemize}
    \item \al{Macro}: dynamic general equilibrium models, heterogeneous agents, ...
    \item \al{Micro}: dynamic games, life-cycle models, industry dynamics, ...
    \item \al{Econometrics}: machine learning, non-standard estimators, large datasets, ...
    \item \al{International/spatial}: models with heterogeneous firms and countries, dynamic models of trade, spatial models, climate change, ...
    \item \al{Finance}: asset pricing, risk, non-arbitrage conditions, ...
    \item \al{Economic history}: large sets of non-standard information, library data, historical counterfactuals, ...
\end{itemize}
\item Judd (1997): ``Computation \alg{helps}, \alb{complements}, and \alr{extends} economic and econometric theory.''
\end{itemize}
\end{frame}


\begin{frame}{Quantitative economics}

    \alb{Loosely:} a study that solves and estimates structural models using computational techniques.
    
    \vspace{0.3cm}
    
    \begin{itemize}
        \item \al{Question}: measurement.
        \item \al{Answer}: numbers.
        \item \al{Key piece}: a structural model (theory of behavior / economy)
        \item Use the model to get quantitative implications of the theory.
        \item The model is \alb{calibrated} along some dimensions and used to explain some
        other dimensions of the data.
        \item The computer is used to solve the model and run
        computational experiments to answer the research question (and explain
        mechanism behind the result).
    \end{itemize}

\end{frame}

\begin{frame}{Example - investment subsidy}
    Suppose we are interested in the effect of an investment subsidy on firm investment behavior:
    \begin{itemize} 
        \item Build a model with profit-maximizing firms that differ in size, sales, employment, and productivity as in the data.
        \item The investment behavior of firms crucial for this research question. We will choose parameters of the model to match this behavior. Note: some parameters might not correspond directly to what we see in the data - by bringing the model close to the data we \al{learn} about their values, 
        \item We now have a laboratory that can help us with the research question. We compare two versions: with and without the subsidy. 
        \item We can examine which features of the model matter the most for the result. Or how the subsidy is introduced (anticipated, non-anticipated). 
    \end{itemize}
\end{frame}

\begin{frame}{Example - investment subsidy}
    \begin{itemize}
     \item Problem of a firm:  \begin{align*}
\max_{\{k_{t+1},l_t,i_t\}_{t=0}^{\infty}} \mathbb{E}_0 \left[ \sum_{t=0}^{\infty} \frac{1}{(1+r)^t} \left( p f\of{z_t,k_t,l_t} - wl_t - (1-\tau) p^I i_t - c\of{k_{t+1},k_t,i_t} \right) \right], \\ 
        \text{s.t.} \ k_{t+1} = (1-\delta) k_t + i_t, \quad z_{t+1} \sim Q\of{z_t,\cdot}, \quad \forall t \geq 0, \quad k_0, z_0 \ \text{given}.
  \end{align*}
     \item We will learn how to use \al{dynamic programming} to solve such problems. 
    \end{itemize} 
    \end{frame}

\begin{frame}{Example - investment subsidy}
    \begin{itemize}
     \item Recast it as \begin{align*}
        V\of{k,z} = \max_{k',l,i} \mleftright{ p f\of{z,k,l} - wl - (1-\tau) p^I i - c\of{k',k,i} + \frac{1}{1+r} \int V\of{k',z'} Q\bp{z,dz'} }, \\ \text{s.t.} \ k' = (1-\delta) k + i.
     \end{align*} 
        \item Solve it to get policy functions such as $i\of{k,z}$. 
        \item We now know how much capital next period $k'$ chooses a firm with $\bp{k,z}$. Use it together with the stochastic process for $z$ to track the distribution of firms over $\bp{k,z}$, $\mu_t\of{k,z}$.
        \item We might be interested in a stationary distribution $\mu \of{\cdot} = \mu_{t} \of{\cdot} = \mu_{t-1} \of{\cdot}$
        \item We will learn how to find it. 
    \end{itemize}
\end{frame}


\begin{frame}{Example - investment subsidy}
    \begin{itemize}
         \item How to parametrize the model? For example, let $$\log{z'} = \rho \log{z} + \epsilon', \ \epsilon' \sim N\of{0,\sigma^2}.$$ What values of $\rho$ and $\sigma^2$ make sense?  
        \item Calculate various statistics using $\mu\of{k,z}$ and policy functions. Match them to the data by appropriately choosing parameters of the model.
        \item Vary $\tau$, compare firm behavior, the implied distribution of firms, and the statistics.
        \item We can extend our analysis to consider: \begin{itemize}
        \item endogenous prices $p,w,p^I,r$ that clear some markets (need to model other parts of the economy),
        \item transition between two stationary distributions (need to consider time explicitly),
        \item $\cdots$
        \end{itemize}
    \end{itemize}
\end{frame}


    \begin{frame}{Roadmap}

\begin{enumerate}
    \item Tools
    \begin{itemize}
        \item Introduction to Julia
        \item Numerical methods: root finding, optimization, interpolation
    \end{itemize}

    \item Techniques
    \begin{itemize}
        \item Recursive methods with discrete and continuous states
        \item Projection methods
    \end{itemize}

    \item Economics
    \begin{itemize}
        \item Consumption-savings problems
        \item Search models
        \item Heterogeneous agent models
        \item Dynamic stochastic general equilibrium models
    \end{itemize}
\end{enumerate}
    
\end{frame}

\begin{frame}{Requirements}
    \begin{enumerate}
    \item Problem sets (4) 40\%
    \begin{itemize}
        \item \footnotesize{Up to five students per group. Two weeks for each problem set. Submit code and write-up via GitHub.}
    \end{itemize}
    
    \item Final project 30\%
    \begin{itemize}
        \item \footnotesize{Three weeks to solve it.}
    \end{itemize}
    \item Tests (2) 20\%
    \begin{itemize}
        \item \footnotesize{In class, closed book. Verify if you understood the material and not simply used AI.}
    \end{itemize}
    \item Class participation 10\%
    \begin{itemize}
        \item \footnotesize{Class attendance and participation also rewarded. Sometimes mandatory
        readings, you will be cold-called to give a short (5 minutes) summary of them at
        the beginning of class.}
    \end{itemize}
    \end{enumerate}

\end{frame}

\begin{frame}{Logistics}
    \begin{itemize}
    \item We meet on Wednesdays and Fridays at 9:45, Room B107.
    \item Classes will be a mix of lectures and coding sessions. 
    \item Some classes (mostly practical sessions) will be taught by Marcin Lewandowski and some (mostly lectures) by Piotr Żoch. 
    \item All class materials available on GitHub. 
    \item Problem sets will be graded by Marcin Lewandowski. 
    \item Office hours: by appointment, send us an email. 
    \end{itemize}
\end{frame}

\begin{frame}{Problem Sets}
    \begin{itemize}
    \item Create a GitHub repo for your group. Send us the link to it.
    \item Your group composition \al{must} remain the same throughout the semester.
    \item Submit \al{code} and \al{write-up} via GitHub. 
    \item We will \alr{not} accept submission via email or other means.
    \item Your code \al{must} be in Julia. Your write-up must be in a PDF.  
    \item You can use AI tools to \al{help} you write code, but you \al{must} understand what the code does and be able to explain it. If we have doubts, we will ask you to explain your solution in person. 
    \item We will \alr{not} accept late submissions (unless you have a good reason and let us know in advance).
    \item Your code \al{must} be reproducible. We need to be able to run it without any modifications (except for installing packages through \tt{instantiate}).
    \end{itemize}
\end{frame}


\begin{frame}{Software}
    \begin{itemize}
    \item We will teach you some basics of Julia, but it is practice that makes perfect.
    \item Recommended introduction I: \href{https://juliaacademy.com/p/intro-to-julia}{Julia Academy}
    \item Recommended introduction II: \href{https://julia.quantecon.org/}{QuantEcon}
    \item Amazing book: \href{https://www.manning.com/books/julia-for-data-analysis?utm_source=bkamins&utm_medium=affiliate&utm_campaign=book_kaminski2_julia_3_17_22}{Julia for Data Analysis}
    \item Why Julia? 
    \item My view: debates \emph{"X is better than Y!"} are rather unproductive.
    \end{itemize}   
\end{frame}

\begin{frame}{Software}
    \begin{itemize}
        \item Low-level languages: good performance (C, C++, Fortran)
        \item High-level languages: good productivity (Mathematica, Matlab, R, Python)
        \item Julia: good performance and productivity
        \begin{itemize}
            \item Modern language.
            \item High performance and easy to parallelize.
            \item Easy to use.
        \end{itemize}
        \item In quant. economics you will mostly see Fortran, Matlab, Julia and Python.
        \item Mathematica is very useful for symbolic algebra.
        \item Good to know more than one (+ maybe something like R/Stata).
        \item Once you know one, it is easy to learn another. Especially with AI. 
    \end{itemize}   
\end{frame}
\end{document}