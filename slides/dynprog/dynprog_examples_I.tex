\documentclass[11pt,xcolor={dvipsnames},aspectratio=159,hyperref={pdftex,pdfpagemode=UseNone,hidelinks,pdfdisplaydoctitle=true},usepdftitle=false]{beamer}
\usepackage{presentation}[aspectratio=169]
\usepackage{math}
\usepackage{mathtools}
\usepackage{mleftright}
\usepackage{algorithm}% http://ctan.org/pkg/algorithms
\usepackage{algpseudocode}% http://ctan.org/pkg/algorithmicx
\hypersetup{
    colorlinks=magenta,
    linkcolor=magenta,
    filecolor=magenta,      
    urlcolor=magenta,
    }
% Enter title of presentation PDF:
\hypersetup{pdftitle={Intro to Dynamic Programming: Examples}}


\begin{document}
% Enter presentation title:
\title{Intro to Dynamic Programming: Examples}
\subtitle{Quantitative Economics 2024}
% Enter presentation information:

% Enter presentation authors:
\author{Piotr Żoch}%
% Enter presentation location and date (optional; comment line if not needed):
\frame{\titlepage}

% Fill out content of presentation:
\begin{frame}{Introduction}   
    \begin{itemize}
        \item Shortest Path
        \item Tree-cutting
        \item Resource Extraction
        \item Neoclassical Growth Model  
    \end{itemize}
    \end{frame}
    
    \begin{frame}
        \heading{Shortest Path}
        \end{frame}
    

\begin{frame}{Shortest Path}   
    \begin{itemize}
    \item  We have a set of nodes, $S$: $s_1, s_2 \ldots, s_{N}$.
    \item  Let $F_s$ be a set of nodes that can be reached from the node $s$ in one step. 
    \item  $c\of{s,s'}$ is the cost of moving from node $s$ to node $s'\in F_s$.
    \item  The goal is to find the shortest path from some starting node to some destination $d$.
\end{itemize}
\end{frame}

\begin{frame}{Shortest Path}   
    \begin{itemize}
    \item Let $J\of{s}$ be the length of the shortest path from the node $s$ to the destination. Suppose we know $J\of{s}$ for all $s\in N$.
    \item Suppose the starting node is $s$. We can find the shortest path by choosing a node $s'$ such that \begin{align*}
        s' = \argmin_{s'\in F_{s}} c\of{s,s'} + J\of{s'}.
    \end{align*}
    \item It must also be the case that \begin{align*}
        J\of{s} = \min_{s'\in F_{s}} c\of{s,s'} + J\of{s'}
    \end{align*}
    \item The above is the \al{Bellman equation} for the shortest path problem.
    \item \alb{State}: current node, $s$.
\end{itemize}
\end{frame}
 

\begin{frame}{Shortest Path}   
    \begin{itemize}
    \item  If we know $J\of{s}$ for all $s\in S$, we can find the shortest path easily. But we don't know it! 
    \item  How to find $J$? We only know that $J\of{d} = 0$. 
    \item  We will start with a guess $J_0\of{s}$ for all $s\in N$: \begin{itemize} 
        \item $J_0\of{s} = M$ for all $s\neq d$, where $M$ is a very large number;
        \item $J_0\of{s} = 0$.
    \end{itemize}
\end{itemize}
\end{frame}

\begin{frame}{Shortest Path}   
    \begin{itemize}
    \item We will use the following algorithm: \begin{enumerate}
    \item Set $k=0$.
    \item Set $J_{k+1}\of{s} = \min_{s'\in F_s} c\of{s,s'} + J_k\of{s'}$ for all $s\in N$.
    \item If $J_{k+1} = J_k$, stop. Otherwise, set $k = k+1$ and go to the previous step.
    \end{enumerate}

    \end{itemize}
\end{frame}


\begin{frame}
    \heading{Tree-cutting}
\end{frame}


\begin{frame}{Tree-cutting}   
    \begin{itemize}
    \item  There is a tree of size $s$. 
    \item  You can either \alr{cut down}  the tree now and sell the wood or \alb{wait} until next period: \begin{itemize} 
    \item  If you \alr{cut down}  the tree now, you get $f(s)$ dollars. You do not have the tree anymore. 
    \item  If you \alb{wait} until the next period, the tree grows to size $s+h$, unless $s = \bar{S}$, which is the maximum size of the tree. If $s=\bar{S}$ the tree stays the same size.
    \end{itemize}
    \item  Your objective function is to maximize the present discounted value of money you get from cutting down the tree. 
    \item  The discount factor is $\frac{1}{1+r}$, where $r>0$ is the interest rate.
\end{itemize}
\end{frame}

\begin{frame}{Tree-cutting}   
    \begin{itemize}
    \item Let $v\of{s}$ be the value function. 
    \item \alb{State}: the size of the tree, $s$.
    \item Bellman equation: \begin{align*}
        v\of{s} = \max\bc{\underbrace{f\of{s}}_{\alr{\text{cut down}}}, \underbrace{\frac{1}{1+r} v\of{\min{\bar{S},s+h}}}_{\alb{\text{wait}}}}.
    \end{align*}
    \item We can use value function iteration to solve this problem. 
    \end{itemize}
\end{frame}


\begin{frame}{Tree-cutting}   
    \begin{enumerate}
    \item Suppose that cutting down the tree costs $c>0$ dollars. 
    \item Suppose that if you wait until the next period, the tree might not grow at all - it happens with probability $p$.
    \item Suppose that if you wait until the next period, the tree might die which means that you have to cut it down and get only $1/2$ of the value of the tree. It happens with probability $p$.
    \item Suppose that if you wait until the next period, the tree might become sick which means that it will stop growing. It happens with probability $p$. With probability $q$ a sick tree can recover. 
    \end{enumerate}
    \begin{itemize}
    \item How to modify the Bellman equation? 
    \end{itemize}
\end{frame}


\begin{frame}{Tree-cutting}   
    \begin{itemize}
    \item Suppose that cutting down the tree costs $c>0$ dollars. 
    \item Bellman equation: \begin{align*}
        v\of{s} = \max\bc{\underbrace{f\of{s} - c}_{\alr{\text{cut down}}}, \underbrace{\frac{1}{1+r} v\of{\min{\bar{S},s+h}}}_{\alb{\text{wait}}}}.
    \end{align*}
    \end{itemize}
\end{frame}

\begin{frame}{Tree-cutting}   
    \begin{itemize}
    \item Suppose that if you wait until the next period, the tree might not grow at all - it happens with probability $p$.
    \item Bellman equation: \begin{align*}
        v\of{s} = \max\bc{\underbrace{f\of{s}}_{\alr{\text{cut down}}}, \underbrace{\frac{1}{1+r} \bs{\bp{1-p} \cdot v\of{\min{\bar{S},s+h}} + p \cdot v\of{s}}}_{\alb{\text{wait}}}}.
    \end{align*}
    \end{itemize}
\end{frame}

\begin{frame}{Tree-cutting}   
    \begin{itemize}
        \item Suppose that if you wait until the next period, the tree might die which means that you have to cut it down and get only $1/2$ of the value of the tree. It happens with probability $p$.
    \item Bellman equation: \begin{align*}
        v\of{s} = \max\bc{\underbrace{f\of{s}}_{\alr{\text{cut down}}}, \underbrace{\frac{1}{1+r} \bs{\bp{1-p} \cdot v\of{\min{\bar{S},s+h}} + p \cdot \frac{1}{2} f\of{s}}}_{\alb{\text{wait}}}}.
    \end{align*}
    \end{itemize}
\end{frame}

\begin{frame}{Tree-cutting}   
    \begin{itemize}
        \item Suppose that if you wait until the next period, the tree might become sick which means that it will stop growing. It happens with probability $p$. With probability $q$ a sick tree can recover. 
        \item Additional state variable - health of the tree. 
        \item Bellman equation ($H$ is for healthy, $S$ is for sick): \begin{align*}
        v^H\of{s} &= \max\bc{\underbrace{f\of{s}}_{\alr{\text{cut down}}}, \underbrace{\frac{1}{1+r} \bs{\bp{1-p} \cdot v^H\of{\min{\bar{S},s+h}} + p \cdot \cdot v^S\of{s}}}_{\alb{\text{wait}}}} \\
        v^S\of{s} &= \max\bc{\underbrace{f\of{s}}_{\alr{\text{cut down}}}, \underbrace{\frac{1}{1+r} \bs{\bp{1-q} \cdot v^S\of{s} + q \cdot v^H\of{\min{\bar{S},s+h}}}}_{\alb{\text{wait}}}} 
    \end{align*}
    \end{itemize}
\end{frame}

\begin{frame}
    \heading{Resource Extraction}
\end{frame}


\begin{frame}{Resource Extraction}
    \begin{itemize}
    \item You have a resource of size $s$ that you can extract. $s$ is an integer. 
    \item The current price of the resource is $p \iid \phi$ with $N$ possible values: $p_1, p_2, \ldots, p_N$. 
    \item The cost of extracting $x$ units of the resource is $c\of{x}$. Extracting $x$ units of the resource reduces the size of the resource to $s-x$ units next period. It is never possible to extract more than $s$ units of the resource.
    \item The objective is to maximize the expected present discounted value of the profit from extracting the resource:
    \begin{align*}
        \E_0 \sum_{t=0}^\infty \b^t \bp{p_t x_t - c\of{x_t}} 
    \end{align*}
\end{itemize}
\end{frame}


\begin{frame}{Resource Extraction}
    \begin{itemize}
    \item \alb{State}: the size of the resource, $s$, and the price of the resource, $p$.
    \item Bellman equation: 
    \begin{align*}
        v\of{s,p} = \max_{x\in\bc{0,1,\ldots,s}} \bc{p x - c\of{x} + \b \sum_{p^\prime \in \text{P}} v\of{s-x,p^\prime} \phi\of{p^\prime}}.
    \end{align*}
\end{itemize}
\end{frame}


\begin{frame}
    \heading{Neoclassical Growth Model}
\end{frame}

\begin{frame}{Neoclassical Growth Model}
    \begin{itemize}
    \item A representative agent's problem \begin{align*}
         \sum_{t=0}^\infty \b^t u\of{c_t}, \quad \text{s.t. } k_{t+1} = f\of{k_t} + \bp{1-\delta} k_t - c_t, \quad k_0 \text{ given}.
    \end{align*}
    \item $\beta{\in}\bp{0,1}$ is the discount factor, $c_t\geq0$ is the consumption, $k_{t}\geq 0$ is the stock of capital.
    \item $u\of{\cdot}$ is the utility function, $f\of{\cdot}$ is the production function, $0<\delta\leq 1$ is the depreciation rate.
    \item We will assume $u\of{c} = \frac{c^{1-\gamma}}{1-\gamma}$, $f\of{k} = k^\alpha$, with $\gamma>0,\alpha\in\bp{0,1}$.
\end{itemize}
\end{frame}

\begin{frame}{Neoclassical Growth Model}
    \begin{itemize}
        \item \alb{State}: the stock of capital, $k$.
        \item Bellman equation: 
        \begin{align*}
            v\of{k} = \max_{c\in \bs{0,f\of{k} + \bp{1-\delta} k}} \bc{u\of{c} + \b v\of{f\of{k} + \bp{1-\delta} k - c}}.
        \end{align*}
        \item Alternatively, we can write it as:
        \begin{align*}
            v\of{k} = \max_{k'\in \bs{0, f\of{k}+\bp{1-\delta}k}} \bc{u\of{f\of{k} + \bp{1-\delta} k - k'} + \b v\of{k'}}.
        \end{align*}
\end{itemize}
\end{frame}

\begin{frame}{Neoclassical Growth Model}
    \begin{itemize}
       \item  Note: in the problem itself there is nothing that says there is only a finite number of $k$.
       \item We will discretize the state space and create a finite grid of $k$: $k_1, k_2, \ldots, k_N$.
       \item We need to decide on the grid size and the grid points. These are properties of the solution method, not the problem itself! 
       \item What we will find is only an approximation to the true value and policy function.
\end{itemize}
\end{frame}

\begin{frame}{Neoclassical Growth Model}
    \begin{itemize}
       \item Let $k^*$ be a level of capital that satisfies \begin{align*}
        1 = \beta \bs{\alpha {k^*}^{\alpha-1} + 1-\delta}.
       \end{align*}
       \item In this model we call $k^*$ the \al{steady state} level of capital. If $k_t = k^*$, then $k_{t+1} = k^*$.
       \item We also know that if we start from $k_0>0$, then $k_t \to k^*$ as $t\to\infty$.
       \item It makes sense to have a grid that is located around $k^*$.   
\end{itemize}
\end{frame}

\end{document}